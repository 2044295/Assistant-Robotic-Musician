\documentclass[11pt, twoside]{article}
  \usepackage[margin=1.25in]{geometry}
  \usepackage{graphicx}
  \usepackage{indentfirst}
  \providecommand{\tightlist}{\setlength{\itemsep}{0pt}\setlength{\parskip}{0pt}}
  \setlength{\fboxsep}{0pt}
  \setlength{\fboxrule}{1pt}

\title{\textsc{Solving the 8-Peice Soma Cube}}
\author{\textsc{Evan Keeton}}
\date{}

\begin{document}
  \maketitle

  \section{Disasembly}
    \subsection{}
      Remove the plastic casing from the cube, and spread out the pieces.
        Arrange the pieces as shown in the next section.

  \section{Preparation}
    \begin{center}
      \fbox{\includegraphics[height=5cm]{./Images/0_Start0.jpg}}
    \end{center}

    Before starting, make sure that your pieces match the pieces in the picture. This is essential for
      following the upcoming steps. Also, be sure that you keep track of which pieces are which. For the
      duration of this tutorial, the pieces will be named by their numbers; labeled from left to right,
      top to bottom, the pieces are as follows:

    \begin{enumerate}
      \tightlist
      \item \texttt{Piece \#1} with a right angle at one end of the ``spine,'' forming an ``L''.
      \item \texttt{Piece \#2} with the ``non-spinal'' piece shifted centrally, forming a ``T''.
      \item \texttt{Piece \#3} with a 3-cube right angle and a non-coplanar cube on the central piece.
      \item \texttt{Piece \#4} with three cubes forming a right angle.
      \item \texttt{Piece \#5}, a $2\times2$ square with the top row shifted to the left, forming a ``snake.''
      \item \texttt{Piece \#6}, a 3-cube right angle and one non-coplanar with the shape, on a tip.
      \item \texttt{Piece \#7}, identical to \texttt{Piece \#4}.
      \item \texttt{Piece \#8}, a single cube.
    \end{enumerate}

  \clearpage

  \section{Solution}
    \subsection{}
      Place \texttt{Piece \#1} on the table, with the long section towards you, as shown in the picture.

      \begin{center}
        \fbox{\includegraphics[height=4cm]{./Images/1_Step1.jpg}}
      \end{center}

    \subsection{}
      Place \texttt{Piece \#2} vertically on top of \texttt{Piece \#1}, as shown. The non-colinear should rest
        on \texttt{Piece \#1}, on the cube farthest (3 cubes) from the right angle.

      \begin{center}
        \fbox{\includegraphics[height=4cm]{./Images/2_Step2.jpg}}
      \end{center}

    \subsection{}
      Place \texttt{Piece \#3} so that one of its non-central cubes fills the hole between the ``L-tip''
        of \texttt{Piece \#1} and the spine of \texttt{Piece \#2}. The other two non-central cubes should
        rest on two of the cubes remaining in \texttt{Piece \#1}, not touching the cube at the angle of the
        shape, as shown in the picture.

      \begin{center}
        \fbox{\includegraphics[height=4cm]{./Images/3_Step3.jpg}}
      \end{center}

    \clearpage
    \subsection{}
      Place one of the exterior cubes of \texttt{Piece \#4} on top of the remaining cube of \texttt{Piece \#1}
        as shown in the picture. The central and third cube of \texttt{Piece \#4} should be towards you, and
        the third cube should be next to the cube in \texttt{Piece \#1}.

      \begin{center}
        \fbox{\includegraphics[height=4cm]{./Images/4_Step4.jpg}}
      \end{center}

    \subsection{}
      Place \texttt{Piece \#5} into the shape with the single cube in the third layer of \texttt{Piece \#2}
        in the hole of the upper-right-hand corner of \texttt{Piece \#5}, the place where a cube would be if
        \texttt{Piece \#5} was a perfect square. Use the picture as a reference.

      \begin{center}
        \fbox{\includegraphics[height=4cm]{./Images/5_Step5.jpg}}
      \end{center}

    \subsection{}
      Place \texttt{Piece \#6} into the shape with two of its pieces on top of \texttt{Piece \#5}. The other
        two pieces should hang down on the right of \texttt{Piece \#5}. Your shape should match the shape in
        the image.

        \begin{center}
          \fbox{\includegraphics[height=4cm]{./Images/6_Step6.jpg}}
        \end{center}

    \clearpage
    \subsection{}
      Place \texttt{Piece \#7} vertically---as shown in the image---with one cube underneath the section of
        \texttt{Piece \#6} extending downards. The other two cubes of \texttt{Piece \#7} should extend up next
        to the aforementioned section of \texttt{Piece \#6}.

      \begin{center}
        \fbox{\includegraphics[height=4cm]{./Images/7_Step7.jpg}}
      \end{center}

    \subsection{}
      Place the remaining piece---\texttt{Piece \#8}---into the singular remaining hole in the shape. This
        piece should be on top of the vertical section of \texttt{Piece \#7}. If done correctly, your cube
        should match the cube shown in the following picture.

      \begin{center}
        \fbox{\includegraphics[height=4cm]{./Images/8_Step8.jpg}}
      \end{center}

    \subsection{}
      Celebrate! You've completed your cube!

      \begin{center}
        \fbox{\includegraphics[height=4cm]{./Images/8_Step8.jpg}}
      \end{center}

\end{document}
